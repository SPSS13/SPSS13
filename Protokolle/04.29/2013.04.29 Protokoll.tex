\documentclass{scrartcl}

\usepackage[en,break]{ukon-infie}

\Names{L. Kr\"amer}
\Term{SS 13}
\Lecture[SWP]{Softwareprojekt}
\date{30.04.2013}
\title{Protokoll 2. Treffen Montag 29.04. 13:00}
\def\oder{\vee}
\def\und{\wedge}

\begin{document}
\maketitle
\section{Ort}
V-Pool (V304), 13:00-13:30
\section{Anwesenheit}

Leonard Krämer, Matthias Kraus, Rebecca Kehlbeck, Fabian Vollmer \\
Später: Nicolas Siebeck, Matthias Miller \\
Entschuldigt: Philipp Hilpert
\section{Agenda}
\begin{itemize}
\item Letztes Treffen
\item Teamname
\item Teamregeln
\item Git-Tutorial
\item Tasks
\item Ausblick
\item Ideen Fragen und sonstige TO DO's
\end{itemize}
\section{Letztes Treffen}
Protokoll fehlt, Leonard Krämer wurde gesagt was erarbeitet wurde
\section{Teamname}
Vorschläge: Perfektes Matching, Graph Maga \\
Abstimmung von Matthias Kraus, Rebecca Kehlbeck, Fabian Vollmer und Leonard Krämer ergab einstimmig 'Graph Maga' als Teamname.
\emph{Zusatz 30.04.: Alle anderen Teammitglieder haben mittlerweile zugestimmt.}
\section{Teamregeln}
Wir haben uns auf folgende Regeln verständigt:
\begin{itemize}
\item Respektvoller Umgang untereinander
\item Pünktlichkeit
\item ASAP an doodlen teilnehmen
\item Frühzeitige Abmeldung, wenn man ausfällt
\item Versionskontrolle verwenden, keine Code-Dateien in der Dropbox oder auf Drive
\item Rückmeldung, wenn Tasks per Email kommen, an den Teamleiter
\item Tasks zügig bearbeiten, wenn man merkt, dass die Zeit nicht ausreicht umgehend beim PM melden
\item Wochenenden sind so weit es geht frei (werden nicht als Projektzeit verplant)
\item Whatsapp Gruppe nur für Terminabsprachen und wenn es brennt
\item Agreedo verwenden
\item Email täglich checken
\item Emails Taggen: [SWP] - topic
\end{itemize}
\section{Git-Tutorial}
War sehr aufschlussreich, die nötigen Dateien und die Präsentation finden sich hier:\\ https://drive.google.com/?authuser=0\#folders/0By3ytU2yKjRZUzNEREg4bnowTjQ
Rebecca macht uns alle zu Editoren des Github Accounts SPSS13
\section{Tasks}
Matthias Miller: Erstellung eines Inhaltsverzeichnisses für das SRS \\
Matthias Kraus, Nico Siebeck, Fabian Vollmer: Fragen für die Requirements erarbeiten.
\section{Ausblick}
Wöchentliches Meeting wird per Doodle entschieden\\
Milestones im Projektplan \\
Projektplan fertig machen fertig und einreichen. \\
Projektplanpräsentation nächste Woche. \\
Concall mit Herrn de Ridder. \\
Nach den Requirements gleich das SRS anfangen. \\
Treffen bei den Projektplanpräsentationen.
\section{Ideen Fragen und sonstige TO DO's}
N/A
\end{document}