\documentclass{scrartcl}

\usepackage[en,break]{ukon-infie}

\Names{L. Kr\"amer}
\Term{SS 13}
\Lecture[SWP]{Softwareprojekt}
\date{21.05.2013}
\title{Protokoll 3. 'Weekly'-Meeting Dienstag 16.05. 17:00}
\def\oder{\vee}
\def\und{\wedge}

\begin{document}
\maketitle
\section{Ort und Zeit}
V-Pool (V304), 17:00-17:37 \\
\section{Anwesenheit}
Leonard Krämer, Matthias Kraus, Rebecca Kehlbeck, Fabian Vollmer, Matthias Miller, Philipp Hilpert, Stephan Heidinger, Nicolas Siebeck
\section{Agenda}
\begin{itemize}
\item Abgeschlossene Tasks
\item Planung nächste Wochen
\item Neue Tasks
\item Ideen Fragen und sonstige TO DO's
\end{itemize}

\section{Abgeschlossene Tasks}
\paragraph{SRS} Erste Revision kam gut an
\section{Planung nächste Wochen}
Der Arbeitsaufwand wurde vom Kunden auf 6 Wochen geschätzt, darum versuchen wir so bald wie möglich mit der Implementation zu beginnen.\\
Wir beginnen plangemäß mit dem SDD und versuchen dabei möglichst effizient zu arbeiten um für die Implementation Zeit zu haben. \\
Das SDD wird folgendermaßen erstellt:
\begin{itemize}
\item Beschreibung der bestehenden Schnittstellen, bis 24.05.
\item Unsere Vorgeschlagene Architektur, bis 28.05.
\item Einleitung, Hardware/Software-Mapping, bis 30.05.
\item Tests, bis 04.06.
\end{itemize}

\section{Tasks}
\paragraph{SDD} Wir beginnen das SDD, damit das vorhandene System zu beschreiben, dabei werden drei Teile getrennt betrachtet. Da für uns besonders die Schnittstellen zwischen den Bibliotheken interessant sind analysieren wir diese. \begin{itemize}
\item ISGCI - Jgraph: Matthias Miller, Nicolas Siebeck
\item Jgraph - JgraphT: Fabian Vollmer, Rebecca Kehlbeck
\item ISGCI - JgraphT: Philipp Hilpert, Matthias Kraus
\end{itemize}
Leonard Krämer hilft aus, wo es nötig ist.

\paragraph{GIT Präsentation} Präsentation mit Thorsten Sauter, Rebecca übernimmt die Präsentation am 28.05.
\paragraph{SRS} Das SRS sieht gut aus, Herr de Ridder hat es sich angeschaut und ein paar kleinere Punkte bemängelt. Diese wurden verbessert und am Mittwoch 22.05 - 13:30 werden diese besprochen.
\section{Ideen Fragen und sonstige TO DO's}
\paragraph{Protokolle}Protokolle einen Tag vor Abgabe nochmal rumschicken - durchlesen, Rechtschreibfehler dem Autor senden. Wenn es keiner liest, sind alle verantwortlich.
\paragraph{Drive Abonnieren}Alle abonnieren auf Google Drive die beiden Tabellen Tasks und Overview, um den Fortschritt des Projektes besser überblicken zu können.
\end{document}