\documentclass{scrartcl}

\usepackage[en,break]{ukon-infie}

\Names{L. Kr\"amer}
\Term{SS 13}
\Lecture[SWP]{Softwareprojekt}
\date{21.05.2013}
\title{Protokoll 4. 'Weekly'-Meeting Donnerstag 23.05. 17:00}
\def\oder{\vee}
\def\und{\wedge}

\begin{document}
\maketitle
\section{Ort und Zeit}
V-Pool (V304), 17:00-17:37 \\
\section{Anwesenheit}
Leonard Krämer, Matthias Kraus, Rebecca Kehlbeck, Fabian Vollmer, Matthias Miller, Philipp Hilpert, Nicolas Siebeck \\ Entschuldigt: Stephan Heidinger
\section{Agenda}
\begin{itemize}
\item Abgeschlossene Tasks
\item Planung nächste Wochen
\item Neue Tasks
\item Ideen Fragen und sonstige TO DO's
\end{itemize}

\section{Abgeschlossene Tasks}
\paragraph{Meeting mit Herr de Ridder}
Philipp Hilpert und Leonard Krämer berichteten dem Rest des Teams, wie das Treffen verlief und klärten einige Fragen. \\
Bemerkenswertes: In der bisherigen Implementation wird JGraph nicht verwendet, darum waren die meisten unserer Anstrengungen der letzten Woche nutzlos, da wir es vergeblich gesucht haben.
\paragraph{SRS} Weitere Änderungen wurden eingepflegt
\section{ Planung nächste Wochen}
Das nächste Treffen wird am 28.05. um 11:35 stattfinden
\section{Tasks}
\paragraph{SDD} Beschreiben der Klassen von ISGCI und des GUI in Gruppen. Wichtige Klassen ausführlicher. GUI Package: Fabian Vollmer, Rebecca Kehlbeck, Nicolas Siebeck; Layout Package: Matthias Miller; jgrapht Package: Leonard Krämer, Matthias Kraus. \\

\paragraph{GIT Präsentation} Präsentation mit Thorsten Sauter, Rebecca übernimmt die Präsentation am 28.05.

\section{Ideen Fragen und sonstige TO DO's}
\paragraph{neues Design für zusätzliche Informationen} Wir haben das neue Design besprochen und werden dafür Mockups erstellen um mit Herrn de Ridder darauf einzugehen. Es wird ein zusätzliches Feld am Bildrand geben, um Informationen, wie alternative Namen, oder Definitionen der Graphklassen anzuzeigen.
\end{document}